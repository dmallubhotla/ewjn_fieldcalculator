\documentclass[11pt]{article}

\usepackage{graphicx}
\usepackage{amsmath}
\usepackage{amssymb}
\usepackage{physics}
\usepackage[plain]{fancyref}

\title{Long Skin Depth spherical metal values}
\date{}
%\author{Deepak Mallubhotla}
\author{}

\begin{document}

\graphicspath{{figures/}}

\maketitle

\section{Approximating using a dipole source}\label{sec:spheredipolesource}

Three plots of $B_z(x, 0, 0)$ induced for a metal sphere of radius $1$ wih skin depth $\delta = 250$ for a uniform source potential as well
as a dipole source potential, with dipole located at $(x, 0, 0)$.
The uniform source potential is set so that for each $x$ it has the same field as the dipole source.
Both are compared against the expected induced field for a metal sphere $B_z = \frac{1}{15 x^6 \delta^2}$.

\begin{figure}[htp]
	\centering
	\includegraphics[width=14cm]{bZAlongXplot1to5}
	\caption{$B_z$ as a function of $x$, for small $x$ \label{fig:bZZplotx1to5}}
\end{figure}

For small $x$, as shown in \fref{fig:bZZplotx1to5}, the differences between the dipole source and the uniform source are
apparent.

\begin{figure}[htp]
	\centering
	\includegraphics[width=14cm]{bZAlongXplot5to8}
	\caption{$B_z$ as a function of $x$, where dipole source approaches uniform source \label{fig:bZZplotx5to8}}
\end{figure}

In the regime shown in \fref{fig:bZZplotx5to8} the dipole source effectively acts as a uniform source;
increasing the mesh resolution reduces the error by quite a bit.

\begin{figure}[htp]
	\centering
	\includegraphics[width=14cm]{bZAlongXplot8to10}
	\caption{$B_z$ as a function of $x$, large $x$ \label{fig:bZZplotx8to10}}
\end{figure}

In the large $x$ region in \fref{fig:bZZplotx8to10}, I think precision errors are a more noticeable contributor to the
error, (which might also explain why the result is calculated to be negative).

I also calculated the plots for $B_z(0, 0, z)$ along the $z-$axis, in order to determine both the anisotropy and to
make sure that the problem was correct for other angles.
The uniform source remained consistent, and the calculation for a dipole source gave the expected ratio of $4$,
as graphed in \fref{fig:dipoleSourceAnisotropy}. (I'm guessing that the discrepancy at larger distances is again due to
the same precision errors that threw off the graphs earlier at larger distances).

\begin{figure}[htp]
	\centering
	\includegraphics[width=14cm]{sphereDipoleSourceZXRatio}
	\caption{$\flatfrac{B_z(0, 0, z)}{B_z(x, 0, 0)}$ \label{fig:dipoleSourceAnisotropy}}
\end{figure}

\end{document}